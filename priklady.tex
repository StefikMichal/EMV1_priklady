% !TEX root = priklady.tex

% Přepsáním argumentu na 'false' vypnete balíček 'minted' pro sázení kódu.
% Pro jeho použití lokálně musíte mít v systému dostupný Python 3, python
% knihovnu 'minted' a PDFLaTeX musíte spouštět s argumentem '-shell-escape'.
% Místo něj můžete použít prostředí 'lstlisting'.
\newcommand{\docminted}{false}

% !TEX root = priklady.tex
% https://github.com/VUT-FEKT-IBE/FEKT.tex

\documentclass[
    % Velikost základního písma je 12 bodů
    12pt,
    % Formát papíru je A4
    a4paper,
    % Oboustranný tisk
    twoside,
    % Záložky a metainformace ve výsledném PDF budou v kódování unicode
    unicode,
]{article}

%%%%%%%%%%%%%%%%%%%%
% OBECNÉ NASTAVENÍ %
%%%%%%%%%%%%%%%%%%%%

% Kódování zdrojových souborů
\usepackage[utf8]{inputenc}
% Kódování výstupního souboru
\usepackage[T1]{fontenc}
% Podpora slovenčiny
\usepackage[slovak]{babel}

% Geometrie stránky
\usepackage[
    % Horní a dolní okraj
    tmargin=25mm,
    bmargin=25mm,
    % Vnitřní a vnější okraj
    lmargin=20mm,                   %default 30mm
    rmargin=20mm,
    % Velikost zápatí
    footskip=17mm,
    % Vypnutí záhlaví
    nohead,
]{geometry}

% Zajištění kopírovatelnosti a prohledávanosti vytvořených PDF
\usepackage{cmap}
% Podmínky (pro použití v titulní straně)
\usepackage{ifthen}

%%%%%%%%%%%%%%%
% FORMÁTOVÁNÍ %
%%%%%%%%%%%%%%%

% Nastavení stylu nadpisů
\usepackage{sectsty}
% Formátování obsahů
\usepackage{tocloft}
\setcounter{tocdepth}{1}
% Odstranění mezer mezi řádky v seznamech
\usepackage{enumitem}
\setlist{nosep}
\setitemize{leftmargin=1em}
\setenumerate{leftmargin=1.5em}
\renewcommand{\labelitemi}{--}
\renewcommand{\labelitemii}{--}
\renewcommand{\labelitemiii}{--}
\renewcommand{\labelitemiv}{--}
% Sázení správných uvozovek pomocí '\enquote{}'
\usepackage{csquotes}
% Vynucení umístění poznámek pod čarou vespod stránky
\usepackage[bottom]{footmisc}
% Automatické zarovnání textu k předcházení vdov a parchantů
\usepackage[defaultlines=3,all=true]{nowidow}
% Zalomení části textu pokud není na současné stránce dost místa
\usepackage{needspace}
% Nastavení řádkování
\usepackage{setspace}
\onehalfspacing
% Změna odsazení odstavců
\setlength{\parskip}{1em}
\setlength{\parindent}{0em}

% Bezpatkové sázení nadpisů (vypnute pre styl nadpisov rovnakym fontom ako zvysok dokumentu)
%\allsectionsfont{\sffamily}
% Změna formátování nadpisu a podnadpisů v Obsahu
%\renewcommand{\cfttoctitlefont}{\Large\bfseries\sffamily}
%\renewcommand{\cftsubsecdotsep}{\cftdotsep}

% Použití moderní/aktualizované sady písem
\usepackage{lmodern}

%%%%%%%%%%%
% NADPISY %
%%%%%%%%%%%

\usepackage{titlesec}

\titlespacing*{\section}{0pt}{10pt}{-0.2\baselineskip}
\titlespacing*{\subsection}{0pt}{0.2\baselineskip}{-0.2\baselineskip}
\titlespacing*{\subsubsection}{0pt}{0.2\baselineskip}{-0.2\baselineskip}
\titlespacing*{\paragraph}{0pt}{0pt}{1em}

%%%%%%%%%%
% ODKAZY %
%%%%%%%%%%

% Tvorba hypertextových odkazů
\usepackage[
    breaklinks=true,
    hypertexnames=false,
]{hyperref}
% Nastavení barvení odkazů
\hypersetup{
    colorlinks,
    citecolor=black,
    filecolor=black,
    linkcolor=black,
    urlcolor=blue
}

%%%%%%%%%%%%%%%%%%%%%%%%%%%
% OBRÁZKY, GRAFY, TABULKY %
%%%%%%%%%%%%%%%%%%%%%%%%%%%

% Vkládání obrázků
\usepackage{graphicx}
\usepackage{subfig}
% Nastavení popisů obrázků, výpisů a tabulek
\usepackage{caption}
\captionsetup{justification=centering}
% Grafy a vektorové obrázky
\usepackage{tikz}
\usetikzlibrary{shapes,arrows}
% Složitější tabulky
\usepackage{tabularx}
\usepackage{multicol}
\usepackage{array}

\usepackage{tabularray}
\usepackage{multirow}

\usepackage{float}

\usepackage{pdfpages}

\usepackage{tablefootnote}                  %Poznamky pod ciarou pre tabulky


% Sázení osamocených float prostředí v horní části stránky
\makeatletter
\setlength{\@fptop}{0pt plus 10pt minus 0pt}
\makeatother

% Vynucení vypsání floating prostředí pomocí \FloatBarrier
\usepackage{placeins}

%%%%%%%%%%%%%%
% MATEMATIKA %
%%%%%%%%%%%%%%

% Sázení matematiky a matematických symbolů ('\mathbb{}')
\usepackage{amsmath}
\usepackage{amssymb}
% Sázení fyzikálních veličin
\usepackage{siunitx}
\sisetup{output-decimal-marker = {,}}
\sisetup{range-phrase=\,--\,}  
\sisetup{range-units=single}
\sisetup{exponent-product = \cdot}
%\sisetup{
%display-per-mode = symbol ,     % Jednotky \per v riadku v matematickom mode
%inline-per-mode = symbol        % Jednotky \per v riadku     
%

% Jednotka rovnice na zarovnaná na pravú stranu vedľa čísla pomocou \tagaddtext{}

\makeatletter
\providecommand\add@text{}
\newcommand\tagaddtext[1]{%
  \gdef\add@text{#1\gdef\add@text{}}}% 
\renewcommand\tagform@[1]{%
  \maketag@@@{\llap{\add@text\quad}(\ignorespaces#1\unskip\@@italiccorr)}%
}
\makeatother

\usepackage[version=4]{mhchem}      %Chemicke vzorce

% Sázení řeckých písmen bez kurzívy
\usepackage{upgreek}
%Přeškrtávání v math režimu
\usepackage{cancel}


%%%%%%%%%%%%%%%%%
% ZDROJOVÉ KÓDY %
%%%%%%%%%%%%%%%%%

% Sazba zdrojových kódů
\usepackage[formats]{listings}
% Přepnutí prostředí 'code' do režimu výpisu kódu
\newenvironment{code}{\captionsetup{type=listing}}{}

% Balíček 'minted' budeme používat pouze pokud je jeho hodnota nastavena na 'true'
\providecommand{\docminted}{false}
\ifthenelse{\equal{\docminted}{true}}
{
    % Sazba zdrojových kódů
    \usepackage[newfloat]{minted}
    % Nastavení barev 'minted' kódů
    \usemintedstyle{pastie}
}
{
    % \docminted není 'true', nic neprovádíme
    % Pokud je v dokumentu 'minted' prostředí, dokument se nepodaří přeložit.
}

\IfFileExists{./.repo.tex}{
    % Soubor '.repo.tex' může (re)definovat povinné a nepovinné argumenty
    % souboru 'main.tex'. To lze využít v případech kdy v jednom repozitáři
    % existuje více dokumentů najednou (např. státnicové otázky).
    \input{.repo}
}{}

% Pokud byly nepovinné argumenty zakomentovány nebo vymazány, přidáme prázdné
% definice příkazů, aby bylo dokument možné správně přeložit.
\providecommand{\docdesc}{}
\providecommand{\docgroup}{}
\providecommand{\docurl}{}

\usetikzlibrary{calc}                     % Pokročilé kreslení souřadnic


\newcommand{\predmet}{Elektrotechnické materiály a výrobní procesy I}
\newcommand{\dokument}{Neoficiální sbírka příkladů částí dielektrických materiálů a polovodičů}
\newcommand{\autor}{cinanko}
\newcommand{\verze}{0.5}

%\usepackage{fancyhdr}
%\setlength{\headsep}{5pt}
%\setlength{\headheight}{15pt}
%\addtolength{\topmargin}{-20pt}
%\renewcommand{\sectionmark}[1]{\markright{#1}}
%\fancyhf{}
%\rhead{\fancyplain{}{Prava strana hlavicky}}               %Prava strana hlavicky
%\lhead{\fancyplain{}{Lava strana hlavicky}}                %Lava strana hlavicky
%\cfoot{\fancyplain{}{\thepage}}

\begin{document}

% Tato hlavička pochází ze šablony, kterou vytvořil Ing. Pavel Čudek, Ph.D. pro potřeby předmětů DME a DIZ. 

\begin{titlepage}
	%\begin{tikzpicture}[remember picture, overlay]
	%	\draw[line width = 2pt] ($(current page.north west) + (1cm,-1cm)$) rectangle ($(current page.south east) + (-1cm,1cm)$);
	%\end{tikzpicture}

	\begin{center}
		\vspace{-1.5cm}
		
		\begin{figure}[!h]
			\centering
			\includegraphics[width=0.75\textwidth]{images/UETE_Color_RGB_CZ.png}
		\end{figure}
		
		\vspace{4cm}
	
		\LARGE{\textbf{\predmet}}\\
		\vspace{1cm}

		\textbf{\dokument}\\
		\vspace{0.5cm}

		\large\begin{tabular}{llll}
			Verze: & \verze	& Datum: & \today \\
		\end{tabular}
		
		\vfill

		\color{red}\normalsize
		\textbf{Poznámka:} Nejedná sa o oficiálny výukový materiál, iba pomôcku na učenie zostavenú na základe zápiskov z cvičení. Preto neberiem žiadnu zodpovednosť za chyby v tomto dokumente. Oficiálne riešené príklady sú dostupné v skriptách \break \textbf{Elektrotechnické materiály a výrobní procesy 1} z roku \textbf{2019}. \\
		Príklady sa vzťahujú ku zadaniam z rokov 2023/2024 a 2024/2025, pričom zadania vyzerali byť identické. Všetky práva patria pôvodným autorom výukových materiálov.
		\color{black}

		\Large\begin{tabular}{ll}
			Autor:      & \autor     \\
		\end{tabular}
		\hfill
	\end{center}
\end{titlepage}

%\pagestyle{fancy}

%------------------------------------------------------------

\section*{Vybrané konstanty}

\begin{table}[H]
    \begin{tblr}{
        cells={valign=m, halign=l},
    }
    $c$                 & \SI{2,998e8}{}        & \SI{}{\meter\per\second}          & Rychlost světla                       \\
    $h$                 & \SI{6,626e-34}{}      & \SI{}{\joule\second}              & Planckova konstanta                   \\
    $k$                 & \SI{1,38e-23}{}       & \SI{}{\joule\per\kelvin}          & Boltzmannova konstanta                \\
    $m_a$               & \SI{9,109e-31}{}      & \SI{}{\kilogram}                  & Hmotnost elektronu                    \\
    $m_p$               & \SI{1,672e-27}{}      & \SI{}{\kilogram}                  & Hmotnost protonu                      \\
    $N_A$               & \SI{6,023e-23}{}      & \SI{}{\per\mole}                  & Avogadrova konstanta                  \\
    $n_L$               & \SI{2,688e25}{}       & \SI{}{\per\cubic\meter}           & Loschmidtovo číslo                    \\
    $q$                 & \SI{-1,602e-19}{}     & \SI{}{\coulomb}                   & Náboj elektronu                       \\
    $\varepsilon_0$     & \SI{8,854e-12}{}      & \SI{}{\farad\per\meter}           & Permitivita vakua                     \\
    $\mu_0$             & $4\pi\cdot10^{-7}$    & \SI{}{\henry\per\meter}           & Permeabilita vakua                    \\
    \hline
    $k$                 & \SI{8,617e-5}{}       & \SI{}{\electronvolt\per\kelvin}   & Redukovaná Boltzmannova konstanta     \\
    \end{tblr}
\end{table}

\section*{Vybrané vlastnosti polovodičových materiálů}
Platné při teplotě $T=\SI{300}{\kelvin}$

\begin{table}[H]
    \begin{tblr}{
        cells={valign=m, halign=l},
    }
    Značka      & Křemík                                            & Germánium                                         & Vlastnost                                                                 \\
    $n_i$       & \SI{1,45e16}{\per\cubic\meter}                    & \SI{2,29e19}{\per\cubic\meter}                    & {Koncentrace nosičů proudu (elektronů a děr) \\ ve vlastním polovodiči}   \\
    $W_g$       & \SI{1,11}{\electronvolt}                          & \SI{0,67}{\electronvolt}                          & Šířka zakázaného pásu                                                     \\
    $\mu_n$     & \SI{0,135}{\meter\squared\per\volt\per\second}    & \SI{0,39}{\meter\squared\per\volt\per\second}     & Pohyblivost elektronů                                                     \\
    $\mu_p$     & \SI{0,048}{\meter\squared\per\volt\per\second}    & \SI{0,19}{\meter\squared\per\volt\per\second}     & Pohyblivost děr                                                           \\
    $N_c$       & \SI{2,8e25}{\per\cubic\meter}                     & \SI{1,04e25}{\per\cubic\meter}                    & Efektivní hustota stavů ve vodivostním pásu                               \\
    $N_v$       & \SI{1,04e25}{\per\cubic\meter}                    & \SI{6,0e24}{\per\cubic\meter}                     & Efektivní hustota stavů ve valenčním pásu                                 \\
    \end{tblr}
\end{table}


\newpage

%------------------------------------------------------------

\section{Oblast dielektrických materiálů a izolantů}

\subsection*{1)}
Elektronová polarizovatelnost $\alpha_e$ atomu argonu je \SI{1,43e-40}{\farad\meter\squared}. Určete relativní permitivitu argonu při normálních fyzikálních podmínkách.

% Původní příklad v ručně psané podobě
%\begin{figure*}[h]
%    \centering
%    \includegraphics*[width=0.80\textwidth]{images/diel1.jpg}
%\end{figure*}

$\upalpha_E = 1,43 \cdot 10^{-40} \; \si{\farad\meter\squared}$\\
$\upvarepsilon_V = ?$ \\
\noindent\rule{8cm}{0.4pt}

Claussius - Mossotiho rovnice (C-M):

\begin{equation*}
    \frac{\varepsilon_r - 1}{\varepsilon_r + 2} = \frac{n \cdot \alpha}{3 \cdot \varepsilon_0} \\
\end{equation*}

Argon = inertní plyn $\Rightarrow  \varepsilon_r \approx 1$

\begin{equation*}
    \frac{\varepsilon_r - 1}{1 + 2} = \frac{n \cdot \alpha}{3 \cdot \varepsilon_0}
\end{equation*}

\begin{equation*}
    \frac{\varepsilon_r - 1}{\cancel{3}} = \frac{n \cdot \alpha}{\cancel{3} \cdot \varepsilon_0}
\end{equation*}

\begin{equation*}
    \varepsilon_r= \frac{n \cdot \alpha}{\varepsilon_0} + 1
\end{equation*}

n = n\textsubscript{L} = Loshnidtovo číslo = počet atomů v 1 \si{\cubic\meter} \textbf{plynu} (pro pevné látky jde o hodnotu o 2 až 3 řády větší)

\begin{equation*}
    \varepsilon_r = \frac{n_L \cdot \alpha}{\varepsilon_0} + 1 = \frac{2,688 \cdot 10^{25} \cdot 1,43 \cdot 10^{-40}}{8,854 \cdot 10^{-12}} + 1 = 1,000434 \; [-]
\end{equation*}

\noindent\rule{8cm}{0.4pt}

Vysvětlivky:

\begin{tabular}{rcl}
    $\upalpha$ & = & polarizovatelnost \\
    $\upvarepsilon_0$ & = & permitivita vakua (konstanta) \\
    n & = & počet atomů v 1 \si{\cubic\meter} látky
\end{tabular}

\newpage

%------------------------------------------------------------

\subsection*{2)}
Relativní permitivita $\varepsilon_{rs}$ složeného ze dvou vzájemně nereagujících látek o permitivitách $\varepsilon_{r1}$ a $\varepsilon_{r2}$ se často určuje Lichteneckerovým mocninovým vztahem
\begin{equation}
    {{\varepsilon_{rs}}^k}={V_1{\varepsilon_{r1}}^k}+{V_2{\varepsilon_{r2}}^k}
\end{equation}
v němž $V_1$ a $V_2$ jsou poměrné objemové podíly obou látek a $k$ je empirická konstanta. Hodnota konstanty $k$ se mění v rozsahu $<-1; +1>$ podle tvaru a rozložení částic obou látek; při chaotickém uspořádání částic $k\rightarrow 0$. Ukažte, že v tomto případě přechází mocninový vztah ve vztah logaritmický:
\begin{equation}
    {\log{\varepsilon_{rs}}}={V_1\log{\varepsilon_{r1}}}+{V_2\log{\varepsilon_{r2}}}
\end{equation}

% Původní příklad v ručně psané podobě
%\begin{figure*}[h]
%    \centering
%    \includegraphics*[width=0.75\textwidth]{images/diel2.jpg}
%\end{figure*}

\begin{equation*}
    \varepsilon_{rs}^k = V_1 \cdot \varepsilon_{r1}^k + V_2 \cdot \varepsilon_{r2}^k
\end{equation*}

\begin{equation*}
    \alpha^x = 1 + \underbrace{\frac{x \cdot \ln a}{1!}}_{\textrm{když} \; x \rightarrow 0 \; \textrm{malý příspěvek}} + \overbrace{\frac{x^2 \cdot \ln a^2}{2!}}^{\textrm{řádově ještě měnší}} + \dots 
\end{equation*}

Pokud $x\rightarrow 0$:

\begin{equation*}
    a^x \doteq 1 + \frac{x \cdot \ln a}{1}
\end{equation*}

Na základě toho:

\begin{equation*}
    1 + \frac{k \cdot \ln \varepsilon_{rs}}{1} = V_1 \cdot \left( 1 + \frac{k \cdot \ln \epsilon_{r1}}{1} \right) + V_2 \cdot \left( 1 + \frac{k \cdot \ln \varepsilon_{r2}}{1} \right)
\end{equation*}

V\textsubscript{1} a V\textsubscript{2} jsou objemové podíly, dohromady tedy dají 1

\begin{equation*}
    \cancel{1} + k \cdot \ln \varepsilon_{rs} = \cancel{V_1} + V_1 \cdot k \cdot \ln \varepsilon_{r1} + \cancel{V_2} + V_2 \cdot k \cdot \ln \varepsilon_{r2}
\end{equation*}

k je v každém členu, možno vykrátit

\begin{equation*}
    \cancel{k} \cdot \ln \varepsilon_{rs} = V_1 \cdot \cancel{k} \cdot \ln \varepsilon_{r1} + V_2 \cdot \cancel{k} \cdot \varepsilon_{r2}
\end{equation*}

\begin{equation*}
    \ln \varepsilon_{rs} = V_1 \cdot \ln \varepsilon_{r1} + V_2 \cdot \ln \varepsilon_{r2}
\end{equation*}

A jelikož $\ln x = 2,3 \cdot \log x$:

\begin{equation*}
    \cancel{2,3} \cdot \log \varepsilon_{rs} = V_1 \cdot \cancel{2,3} \cdot \log \varepsilon_{r1} + V_2 \cdot \cancel{2,3} \cdot \log \varepsilon_{r2}
\end{equation*}

\begin{equation*}
    \log \varepsilon_{rs} = V_1 \cdot \log \varepsilon_{r1} + V_2 \cdot \log \varepsilon_{r2}
\end{equation*}

\newpage

%------------------------------------------------------------

\subsection*{3)}
Mezi elektrodami deskového kondenzátoru o rozměrech 7x12 \SI{}{\centi\meter} a vzdálenosti elektrod \SI{5}{\milli\meter} je vložena destička z polystyrenu o tloušťce \SI{3}{\milli\meter}. Zbytek prostoru mezi elekrodami je vyplněn vzduchem za normálních atmosférických podmínek. Vypočtěte kapacitu tohoto kondenzátoru, je-li relativní permitivita polystyrenu při teplotě \SI{20}{\degreeCelsius} rovna 2,3. Jak se změní kapacita kondenzátoru, je-li celý prostor mezi elektrodami vyplněn pěnovým polystyrenem, v němž je objemový podíl polystyrenu a vzduchu stejný jako v prvém případě?

\begin{figure*}[h]
    \centering
    \includegraphics*[width=0.80\textwidth]{images/diel3.jpg}
\end{figure*}

\newpage

\begin{figure*}[h]
    \centering
    \includegraphics*[width=0.85\textwidth]{images/diel3_1.jpg}
\end{figure*}

\newpage

%------------------------------------------------------------

\subsection*{4)}
Rezistivitu elektroizolačních kapalin $\rho_v$ lze v závislosti na teplotě vyjádřit vztahem
\begin{equation}
    \rho=A\cdot\exp^\frac{B}{T}
\end{equation}
v němž $A [\SI{}{\ohm\meter}]$ a $B [\SI{}{\kelvin}]$ jsou materiálové konstanty; teplota T je udána v \SI{}{\kelvin}. Kabelový impregnant složený z minerálního oleje s přídavkem \SI{25}{\percent} (hmotnostních) rafinované kalafuny má při teplotě \SI{20}{\degreeCelsius} rezistivitu \SI{2e10}{\ohm\meter}. Stanovte rezistivitu tohoto impregnantu při teplotách \SI{50}{\degreeCelsius} a \SI{80}{\degreeCelsius}, je li součinitel $B$ roven \SI{7e3}{\kelvin}.

\begin{figure*}[h]
    \centering
    \includegraphics*[width=0.75\textwidth]{images/diel4.jpg}
\end{figure*}

\newpage

%------------------------------------------------------------

\subsection*{5)}
Měřením dynamické viskozity transformátorového oleje BTS2 na Höpplerově viskozimetru byly při několika teplotách zjištěny údaje uvedené v tabulce. Stanovte rezistivitu tohoto oleje při teplotách \SI{50}{\degreeCelsius} a \SI{85}{\degreeCelsius}, je li hodnota rezistivity při teplotě \SI{20}{\degreeCelsius} rovna \SI{3e11}{\ohm\meter}. Při výpočtu předpokládejte, že při změně teploty se nemění koncentrace volných iontů v oleji. \\
\textbf{Tabulka:}
\begin{table}[H]
    \centering
    \begin{tblr}{
        cells={valign=m, halign=c},
        colspec={QQQQQQ},
        columns = 2cm,
        rows = 0.5cm,
        hlines,
        vlines
        }
        $\vartheta[\SI{}{\degreeCelsius}]$                 & 20                & 40                & 60                & 80                & 100               \\
        $\eta[\SI{}{\newton\second\per\meter\squared}]$    & \SI{4,35e-2}{}    & \SI{1,21e-2}{}    & \SI{3,95e-3}{}    & \SI{1,46e-3}{}    & \SI{6,01e-4}{}    \\      
    \end{tblr}
\end{table}

\begin{figure*}[h]
    \centering
    \includegraphics*[width=0.75\textwidth]{images/diel5.jpg}
\end{figure*}

\newpage

\begin{figure*}[h]
    \centering
    \includegraphics*[width=0.85\textwidth]{images/diel5_1.jpg}
\end{figure*}

\newpage

%------------------------------------------------------------

\subsection*{6)}
V obvodu střídavého elektrického proudu je zapojen kondenzátor, jehož dielektrikum vykazuje ztráty. Chování tohoto kondenzátoru lze za předpokladu, že pochody v dielektriku jsou lineární vyšetřit sledováním ekvivalentního dvouprvkového náhradního zapojení kondenzátoru s ideálním, bezztrátovým dielektrikem a odporu představujícího ztráty. Uvažujte, že kondenzátor s ideálním dielektrikem o kapacitě $C_p$ a odpor $R_p$ jsou v náhradním zapojení spojeny paralelně a že je na uvedenou soustavu připojeno napětí $U$. \\
Nakreslete pro tento případ fázorový diagram napětí a proudů soustavy a určete ztrátový činitel, celkovou impedanci a celkové ztráty energie v soustavě.

\begin{figure*}[h]
    \centering
    \includegraphics*[width=0.95\textwidth]{images/diel6.jpg}
\end{figure*}

\newpage

\begin{figure*}[h]
    \centering
    \includegraphics*[width=\textwidth]{images/diel6_1.jpg}
\end{figure*}

\newpage

%------------------------------------------------------------

\subsection*{7)}
Ve smyslu zadání úlohy č. C-7 uvažujte sériové zapojení odporu $R_s$ a kondenzátoru s ideálním dielektrikem $C_s$. K soustavě obou prvků nechť je přiloženo napětí $U$. Nakreslete fázorový diagram napětí a proudů soustavy, určete ztrátový činitel, celkovou impedanci a celkové ztráty energie v soustavě.

\begin{figure*}[h]
    \centering
    \includegraphics*[width=0.85\textwidth]{images/diel7.jpg}
\end{figure*}

\newpage

%------------------------------------------------------------

\subsection*{8)}
Určete ztrátový činitel vzduchu za normálních fyzikálních podmínek a při kmitočtu \SI{50}{\hertz}, má-li rozhodující vliv na velikost ztrát elektrická vodivost vzduchu. Relativní permitivita vzduchu je za normálních fyzikálních podmínek rovna \SI{1,000584}{}, rezistivita je za stejných podmínek \SI{e16}{\ohm\meter}.

\begin{figure*}[h]
    \centering
    \includegraphics*[width=\textwidth]{images/diel8.jpg}
\end{figure*}

\newpage

%------------------------------------------------------------

\subsection*{9)}
Komplexní permitivita $\varepsilon^*$ dielektrika je definována vztahem ${\varepsilon^{*}}={\varepsilon^{'}}-{j\varepsilon^{''}}$. V závislosti na kmitočtu lze podle Debyeho vyjádřit komplexní permitivitu rovnicí
\begin{equation}
    {\varepsilon^{*}}={\varepsilon_\infty}+\frac{{\varepsilon_s}-{\varepsilon_\infty}}{1+j\omega\tau}
    \label{eq:komplex_permitivita_Deby}
\end{equation}
v níž $\varepsilon_s$ značí relativní (statickou) permitivitu dielektrika určenou při kmitočtu $f\rightarrow 0$, $\varepsilon_\infty$ relativní (optickou) permitivitu určenou při velmi vysokých kmitočtech; $\tau$ je relaxační doba, která je mimo jiné i funkcí teploty. \textbf{Vyjděte z obou uvedených vztahů
a určete reálnou část $\varepsilon{'}$ a imaginární část $\varepsilon{''}$ komplexní permitivity}.

Rozdelením rovnice (\ref{eq:komplex_permitivita_Deby}) na reálnu a imaginárnu zložku dostaneme:

\begin{equation*}
    {\varepsilon^{*}}={\varepsilon_\infty}+\frac{{\varepsilon_s}-{\varepsilon_\infty}}{1+j\omega\tau}={\varepsilon_\infty}+\frac{{\varepsilon_s}-{\varepsilon_\infty}}{1+j\omega\tau}\cdot\frac{1-j\omega\tau}{1-j\omega\tau}={\varepsilon_\infty}+\frac{{\varepsilon_s}-{\varepsilon_\infty}}{1+{\omega}^2{\tau}^2}-j \frac{\omega\tau({\varepsilon_s}-{\varepsilon_\infty})}{1+{\omega}^2{\tau}^2}
\end{equation*}

Porovnaním s rovnicou ${\varepsilon^{*}}={\varepsilon^{'}}-{j\varepsilon^{''}}$ dostávame:

\begin{equation*}
    {\varepsilon^{'}}={\varepsilon_\infty}+\frac{{\varepsilon_s}-{\varepsilon_\infty}}{1+{\omega}^2{\tau}^2}=\frac{{\varepsilon_s}+{\varepsilon_\infty}{\omega}^2{\tau}^2}{1+{\omega}^2{\tau}^2} \tag{Reálna časť}
\end{equation*}

\begin{equation*}
    {\varepsilon^{''}}=\frac{\omega\tau({\varepsilon_s}-{\varepsilon_\infty})}{1+{\omega}^2{\tau}^2} \tag{Imaginárna časť}
\end{equation*}

\color{red}Poznámka: Tento príklad \textbf{nebol riešený vrámci cvičenia}. Je prevzatý zo skrípt (BPC-EMV skripta 2019, strana 86 č. 10) a takisto ho možno nájsť aj v prednáškach (Prednáška č. 2, strany~14 - 18). \\

\color{black}

\end{document}

%\SI{}{}