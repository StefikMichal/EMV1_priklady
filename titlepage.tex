% Tato hlavička pochází ze šablony, kterou vytvořil Ing. Pavel Čudek, Ph.D. pro potřeby předmětů DME a DIZ. 

\begin{titlepage}
	%\begin{tikzpicture}[remember picture, overlay]
	%	\draw[line width = 2pt] ($(current page.north west) + (1cm,-1cm)$) rectangle ($(current page.south east) + (-1cm,1cm)$);
	%\end{tikzpicture}

	\begin{center}
		\vspace{-1.5cm}
		
		\begin{figure}[!h]
			\centering
			\includegraphics[width=0.75\textwidth]{images/UETE_Color_RGB_CZ.png}
		\end{figure}
		
		\vspace{4cm}
	
		\LARGE{\textbf{\predmet}}\\
		\vspace{1cm}

		\textbf{\dokument}\\
		\vspace{0.5cm}

		\large\begin{tabular}{llll}
			Verze: & \verze	& Datum: & \today \\
		\end{tabular}
		
		\vfill

		\color{red}\normalsize
		\textbf{Poznámka:} Nejedná sa o oficiálny výukový materiál, iba pomôcku na učenie zostavenú na základe zápiskov z cvičení. Preto neberiem žiadnu zodpovednosť za chyby v tomto dokumente. Oficiálne riešené príklady sú dostupné v skriptách \break \textbf{Elektrotechnické materiály a výrobní procesy 1} z roku \textbf{2019}. \\
		Príklady sa vzťahujú ku zadaniam z rokov 2023/2024 a 2024/2025, pričom zadania vyzerali byť identické. Všetky práva patria pôvodným autorom výukových materiálov.
		\color{black}

		\Large\begin{tabular}{ll}
			Autor:      & \autor     \\
		\end{tabular}
		\hfill
	\end{center}
\end{titlepage}